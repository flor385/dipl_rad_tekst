\documentclass[utf8]{beamer}
\usepackage[utf8]{inputenc}
\usepackage[T1]{fontenc}
\usepackage{booktabs}
% \usecolortheme{crane}

\begin{document}

\title{Jezični model temeljen na neuronskoj mreži}
\author{Florijan Stamenković \\
	Mentor: Marko Čupić}
\institute{Fakultet elektrotehnike i računarstva \\
	Sveučilište u Zagrebu}

\frame{\titlepage}

\begin{frame}
\frametitle{Jezični model}

	\pause

	\begin{block}{Smislene rečenice}
	Petar ide u dućan. Danas je lijep dan.
	\end{block}

	\pause

	\begin{block}{Besmislene rečenice}
	Gavran tanges jučer zelen dva.
	\end{block}

	\pause

	\begin{block}{Matematička (probabilistička) definicija}
	$P(polje | Petar, ide, u) > P(vilica | Petar, ide, u)$ \\
	$P(w_i | w_0, w_1, ..., w_{i - 1})$
	\end{block}

	\pause

	\begin{block}{Aproksimacija korištenjem $(n - 1)$ prethodnih riječi}
	$P(w_i | w_{i - n + 1}, ... , w_{i - 1}) \approx P(w_i | w_0, w_1, ..., w_{i - 1})$
	\end{block}

\end{frame}

\begin{frame}
\frametitle{Jezični model}

	\begin{block}{Primjene}
	\begin{itemize}[<+->]
		\item{Automatsko prevođenje}
		\item{Prepoznavanje govora}
		\item{Ispravka pogreški u pisanju}
		\item{Klasifikacija teksta}
		\item{...}
	\end{itemize}
	\end{block}

\end{frame}

\begin{frame}
\frametitle{Prebrojavanje \textit{n}-grama}

	\begin{block}{Primjerice trigrama}
	\begin{itemize}
		\item{Primjer: Škola je uskoro gotova.}
		\item{Trigrami: (škola je), (je uskoro), (uskoro gotova)}
		\item{$P(w_i | w_{i - 2}, w_{i - 1}) \propto C(w_{i - 2}, w_{i - 1}, w_i)$}
	\end{itemize}
	\end{block}

	\begin{block}{Prednosti i nedostatci}
	\begin{itemize}
		\item{Jednostavan pristup}
		\item{Brz za implementaciju, treniranje i primjenu}
		\item{Rijetkost pojavljivanja onemogućava korištenje velikih $n$}
		\item{Potrebno dobro (Kneser-Ney) zaglađivanje da bi radilo}
	\end{itemize}
	\end{block}

\end{frame}

\begin{frame}
\frametitle{Neuronska mreža}

	\begin{block}{Definicija}
	\begin{itemize}
		\item{Unaprijedna mreža bez skrivenih slojeva}
		\item{Po jedan izlazni neuron za svaku riječ vokabulara}
		\item{Običan klasifikator, riječ je klasa}
		\item{Riječi predočavane $d$-dimenzionalnim vektorima}
	\end{itemize}
	\end{block}

	\begin{block}{Prednosti i nedostatci}
	\begin{itemize}
		\item{Veliki parametarski prostor (veličina vokabulara)}
		\item{Sporo treniranje, osrednji rezultati}
	\end{itemize}
	\end{block}
\end{frame}


\begin{frame}
\frametitle{Log-bilinearni model}

	\begin{block}{Definicija}
	\begin{itemize}
		\item{Riječi predočene $d$-dimenzionalnim vektorima}
		\item{Prethodnih $(n - 1)$ riječ daju $(n - 1) d$ vektor}
		\item{Dobiveni vektor se množi matricom $W$, izlaz je $d$-dimenzionalni vektor}
		\item{Sličnost izlaznog vektora definira vjerojatnost riječi}
	\end{itemize}
	\end{block}

	\begin{block}{Prednosti i nedostatci}
	\begin{itemize}
		\item{Najbolji rezultati, riječ-vektori su nusprodukt}
		\item{Jednostavna formulacija}
		\item{Rezultati samo malo bolji od Kneser-Ney, puno sporije treniranje}
	\end{itemize}
	\end{block}

\end{frame}

\begin{frame}
\frametitle{Ograničeni Boltzmannov stroj}

	\begin{block}{Definicija}
	\begin{itemize}
		\item{Riječi predočene $d$-dimenzionalnim vektorima}
		\item{Jedan skriveni sloj stohastičkih binarnih neurona}
		\item{Energija mreže definira uvjetnu vjerojatnost}
	\end{itemize}
	\end{block}

	\begin{block}{Prednosti i nedostatci}
	\begin{itemize}
		\item{Kompleksno i sporo treniranje stohastičkih neurona}
		\item{Osrednji rezultati}
	\end{itemize}
	\end{block}

\end{frame}

\begin{frame}
\frametitle{Rezultati}

	\begin{block}{Evaluacija mjerom perplexity}
	\begin{itemize}
		\item{$\exp \left( \frac{1}{N} \ln P(w | ...) \right)$}
		\item{Manje je bolje}
		\item{Jako ovisi o vokabularu, korpusu; samo za relativnu usporedbu}
	\end{itemize}
	\end{block}


	\begin{table}[htb]
	\centering
	\begin{tabular}{lccc}
	\toprule
	 & \multicolumn{3}{c}{Parametar \textit{n}} \\
	\cmidrule(r){2-4}
	Model & 3 & 4 & 5 \\
	\midrule
	Additivno &  305 & 1182 & 2680 \\
	Kneser-Ney & 72 & 121 & 204 \\
	Neuronska mreža & 117 & 114 & 113 \\
	Log-bilinear & 102 & 98 & 98 \\
	\bottomrule
	\end{tabular}
	\end{table}

\end{frame}

\begin{frame}
\frametitle{Rezultati}

	\begin{block}{Trajanje treniranja}
	\begin{itemize}
		\item{CPU: Intel i7, 3.5GHz, 4 jezgre, 8 dretvi}
		\item{GPU: Nvidia GTX 960M, 4GB, 640 jezgara}
	\end{itemize}
	\end{block}


	\begin{table}[htb]
	\caption{Trajanje treniranja modela, izraženo u SAT:MINUTE obliku.}
	\label{tbl:eval_time}
	\centering
	\begin{tabular}{lccc}
	\toprule
	 & \multicolumn{3}{c}{Parametar \textit{n}} \\
	\cmidrule(r){2-4}
	Model & 3 & 4 & 5 \\
	\midrule
	Prebrojavanje \textit{n}-grama CPU &  0:01 & 0:01 & 0.01 \\
	Neuronska mreža, GPU & 1:47 & 2:15 & 2:17 \\
	Neuronska mreža, CPU & 13:38 & 13:52 & 14.20 \\
	Log-bilinear GPU & 0:49 & 0:48 & 0:46 \\
	Log-bilinear CPU & 8:07 & 8:01 & 6:56 \\
	\bottomrule
	\end{tabular}
	\end{table}

\end{frame}

\begin{frame}

	Demo i kraj
	
\end{frame}

% etc
\end{document}
